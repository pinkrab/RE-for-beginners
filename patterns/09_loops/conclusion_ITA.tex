% N.B.: \Conclusion{} is a macro name, do not translate
\subsection{\Conclusion{}}

Scheletro di un ciclo da 2 a 9, inclusivo:

\lstinputlisting[caption=x86,style=customasmx86]{patterns/09_loops/skeleton_x86_2_9_optimized_EN.lst}

L'operazione di incremento può essere rappresentata da 3 istruzioni nel codice non ottimizzato:

\lstinputlisting[caption=x86,style=customasmx86]{patterns/09_loops/skeleton_x86_2_9_EN.lst}

Se il corpo del ciclo è piccolo, un intero registro può essere dedicato alla variabile del contatore:

\lstinputlisting[caption=x86,style=customasmx86]{patterns/09_loops/skeleton_x86_2_9_reg_EN.lst}

Alcune parti del loop possono essere generate dal compilatore in ordine diverso:

\lstinputlisting[caption=x86,style=customasmx86]{patterns/09_loops/skeleton_x86_2_9_order_EN.lst}

Solitamente la condizione viene controllata  \IT{prima} del corpo del loop, ma il compilatore potrebbe 
risistemare la logica in modo che la condizione sia controllata \IT{dopo} il corpo del ciclo.

Ciò avviene quando il compilatore è sicuro che la condizione sia sempre vera (\IT{true}) alla prima iterazione,
e di conseguenza che il corpo del loop sia eseguito almeno una volta:

\lstinputlisting[caption=x86,style=customasmx86]{patterns/09_loops/skeleton_x86_2_9_reorder_EN.lst}

\myindex{x86!\Instructions!LOOP}

Uso dell'istruzione \TT{LOOP}. E' rara e i compilatori non la usano. Se la incontrate vi trovate probabilmente di fronte ad
un pezzo di codice scritto a mano:

\lstinputlisting[caption=x86,style=customasmx86]{patterns/09_loops/skeleton_x86_loop_EN.lst}

ARM. 

Il registro \Reg{4} in questo esempio è dedicato interamente alla variabile contatore:

\lstinputlisting[caption=ARM,style=customasmARM]{patterns/09_loops/skeleton_ARM_EN.lst}

% TODO MIPS

