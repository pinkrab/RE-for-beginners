\subsubsection{ARM}

\myparagraph{\NonOptimizingKeilVI (\ARMMode)}

\lstinputlisting[label=Keil_number_sign,style=customasmARM]{patterns/09_loops/simple/ARM/Keil_ARM_O0.asm}

Il contatore di interazione $i$ viene memorizzato nel registro \Reg{4}.
L'istruzione \INS{MOV R4, \#2} inizializza $i$.
Le istruzioni \INS{MOV R0, R4} e \INS{BL printing\_function} rappresentano il corpo del ciclo, in cui la prima istruzione prepara
l'argomento per la funzione \ttf function e la seconda chiama la funzione.
\myindex{ARM!\Instructions!ADD}
L'istruzione  \INS{ADD R4, R4, \#1} aggiunge semplicemente 1 alla variabile $i$ ad ogni iterazione.
\myindex{ARM!\Instructions!CMP}
\myindex{ARM!\Instructions!BLT}
\INS{CMP R4, \#0xA} confronta $i$ con \TT{0xA} (10). 
L'istruzione successiva \INS{BLT} (\IT{Branch Less Than}) salta se $i$ è minotre di 10.
Altrimenti viene scritto 0 nel registro \Reg{0} (poiché la nostra funzione restituisce 0) e l'esecuzione delle funzione termina.

\myparagraph{\OptimizingKeilVI (\ThumbMode)}

\lstinputlisting[style=customasmARM]{patterns/09_loops/simple/ARM/Keil_thumb_O3.asm}

Praticamente uguale al precedente.

\myparagraph{\OptimizingXcodeIV (\ThumbTwoMode)}
\label{ARM_unrolled_loops}

\lstinputlisting[style=customasmARM]{patterns/09_loops/simple/ARM/xcode_thumb_O3.asm}

Questa era la mia funzione \ttf :

\begin{lstlisting}[style=customc]
void printing_function(int i)
{
    printf ("%d\n", i);
};
\end{lstlisting}

\myindex{Unrolled loop}
\myindex{Inline code}
LLVM non solo ha srotolato il loop ma ha anche "messo in linea" (\IT{inlined}) la funzione \ttf e inserito il suo corpo 8 volte
anziché chiamarla.

Ciò puà avvenire quando la funzione è molto semplice (come la mia) e quando non viene chiamata molto spesso (come in questo caso).

\myparagraph{ARM64: \Optimizing GCC 4.9.1}

\lstinputlisting[caption=\Optimizing GCC 4.9.1,style=customasmARM]{patterns/09_loops/simple/ARM/ARM64_GCC491_O3_EN.s}

\myparagraph{ARM64: \NonOptimizing GCC 4.9.1}

\lstinputlisting[caption=\NonOptimizing GCC 4.9.1 -fno-inline,style=customasmARM]{patterns/09_loops/simple/ARM/ARM64_GCC491_O3_EN.s}
