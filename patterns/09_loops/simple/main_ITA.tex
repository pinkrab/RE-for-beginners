\subsection{Simple example}

% subsections
\EN{\input{patterns/09_loops/simple/x86_EN}}
\RU{\input{patterns/09_loops/simple/x86_RU}}
\DE{\input{patterns/09_loops/simple/x86_DE}}
\FR{\input{patterns/09_loops/simple/x86_FR}}
\ITA{\subsubsection{x86}

\myindex{x86!\Instructions!LOOP}

Nel set di istruzioni x86 esiste una speciale istruzione \LOOP che controlla il valore nel registro \ECX, lo decrementa se non è zero
e passa il controllo del flusso alla label specificata nell'operando dell'istruzione.

Probabilmente questa istruzione non è molto conveniente e nessun compilatore moderno la emette automaticamente. Pertanto se trovate questa
istruzione da qualche parte, è molto probabile che si tratti di un pezzo di codice assembly scritto manualmente.


\par

In \CCpp i cicli sono solitamente reaLizzati con i costrutti \TT{for()}, \TT{while()} o \TT{do/while()}.

Iniziamo ad esaminare \TT{for()}.
\myindex{\CLanguageElements!for}

Questo costrutto definisce l'inizializzazione del ciclo (setta il contatore ad un valore iniziale),
la sua condizione (il valore del contatore è più grande di un dato limite?),
cosa avviene ad ogni iterazione (\gls{incremento}/\gls{decremento})
e ovviamente il corpo del ciclo.

\lstinputlisting[style=customc]{patterns/09_loops/simple/loops_1_EN.c}

Anche il codice generato consiste di quattro parti. Iniziamo con l'esaminare un semplice esempio:

\lstinputlisting[label=loops_src,style=customc]{patterns/09_loops/simple/loops_2.c}

Risultato (MSVC 2010):

\lstinputlisting[caption=MSVC 2010,style=customasmx86]{patterns/09_loops/simple/1_MSVC_EN.asm}

Nienete di speciale, GCC 4.4.1 produce pressoché lo stesso codice, con una piccola differenza:

\lstinputlisting[caption=GCC 4.4.1,style=customasmx86]{patterns/09_loops/simple/1_GCC_EN.asm}

Vediamo ora cosa otteniamo una volta abilitata l'ottimizzazione (\TT{\Ox}):

\lstinputlisting[caption=\Optimizing MSVC,style=customasmx86]{patterns/09_loops/simple/1_MSVC_Ox.asm}

In questo caso lo spazione per la variabile $i$ non è più allocato nello stack locale, viene invece usato un registro, \ESI.
Ciò è possibile solo in casi simili, ovvero con piccole funzioni in cui non ci sono molte variabili locali.

Una cosa molto importante è che la funzione \ttf non deve cambiare il valore in \ESI.
Il compilatore ne è sicuro in questo caso.
Se avesse deciso di usare anche il registro \ESI in \ttf, sarebbe stato necessario salvare il suo valore nel prologo della funzione
e ripristinarlo nell'epilogo, quasi come nel nostro listato: si noti \TT{PUSH ESI/POP ESI}
all'inizio e alla fine della funzione.

Proviamo GCC 4.4.1 con il più alto livello di ottimizzazione (\Othree option):

\lstinputlisting[caption=\Optimizing GCC 4.4.1,style=customasmx86]{patterns/09_loops/simple/1_GCC_O3.asm}

\myindex{Loop unwinding}

Oh, GCC ha appena srotolato il loop.

Il \Gls{loop unwinding} rappresenta un vantaggio in tutti i casi in cui non ci sono molte iterazioni ed è possibile guadagnare un po'
di tempo di esecuzione eliminando tutte le istruzioni di supporto al ciclo.
Dall'altra parte, il codice prodotto risulta più lungo.


Cicli molto grandi srotolati non sono raccomandati oggigiorno, poichè funzioni più grandi potrebbero avere un maggiore impatto sulla cache%
%
\footnote{Un ottimo articolo sull'argomento: \DrepperMemory.
Ulteriori raccomandazioni di Intel sul loop unrolling si trovano qui: 
\InSqBrackets{\IntelOptimization 3.4.1.7}.}.

Ok, incrementiamo il valore massimo della variabile $i$ a 100 e proviamo di nuovo. GCC produce:

\lstinputlisting[caption=GCC,style=customasmx86]{patterns/09_loops/simple/2_GCC_EN.asm}

E' molto simile a quanto prodotto da MSVC 2010 con ottimizzazione, con l'eccezione che il registro \EBX viene usato per allocare la variabile $i$.

GCC è sicuro che questo registro non sarà modificato dentro la funzione \ttf, e che in caso contrario verrebbe salvato nel prologo e ripristinato nell'epilogo della funzione,
proprio come nella funzione \main.

\clearpage
\subsubsection{x86: \olly}
\myindex{\olly}

Compiliamo il nostro esempio in MSVC 2010 con le opzioni \Ox e \Obzero, e carichiamolo in \olly.

Sembra che \olly sia in grado di individuare cicli semplici e mostrarli, per convenienza, tra parentesi quadre:

\begin{figure}[H]
\centering
\myincludegraphics{patterns/09_loops/simple/olly1.png}
\caption{\olly: \main begin}
\label{fig:loops_olly_1}
\end{figure}

Tracciando (F8~--- \stepover) possiamo vedere \ESI 
\glslink{incrementare}{incrementare}.
Qui, per esempio, $ESI=i=6$:

\begin{figure}[H]
\centering
\myincludegraphics{patterns/09_loops/simple/olly2.png}
\caption{\olly: corpo del loop appena eseguito, con $i=6$}
\label{fig:loops_olly_2}
\end{figure}

9 è l'ultimo valore del ciclo. Ecco perché \JL non innesca il salto dopo l'\gls{incremento}, e la funzione termina:

\begin{figure}[H]
\centering
\myincludegraphics{patterns/09_loops/simple/olly3.png}
\caption{\olly: $ESI=10$, fine del loop}
\label{fig:loops_olly_3}
\end{figure}

\subsubsection{x86: tracer}
\myindex{tracer}

Come si nota facilmente, non è molto conveniente effettuare il tracciamento manuale nel debugger.
Per questo motivo proviamo ad utilizzare \tracer.

Apriamo l'esempio compilato in \IDA, troviamo l'indirizzo dell'istruzione \INS{PUSH ESI}
(che passa l'unico argomento di \ttf), in questo caso \TT{0x401026} e avviamo il \tracer:

\begin{lstlisting}
tracer.exe -l:loops_2.exe bpx=loops_2.exe!0x00401026
\end{lstlisting}

\TT{BPX} imposta un breakpoint all'indirizzo e \tracer stamperà lo stato dei registri.

Nel file \TT{tracer.log} leggiamo:

\lstinputlisting{patterns/09_loops/simple/tracer.log}

E vediamo come il valore di \ESI cambi da 2 a 9.


Il \tracer può inoltre collezionare i valori dei registri per tutti gli indirizzi all'interno della funzione.
Questo processo è detto \IT{trace}. Ogni istruzione viene "tracciata", e tutti i valori dei registri vengono registrati.

Viene generato uno script \IDA .idc che aggiunge commenti.
Quindi, avendo visto in \IDA che l'indirizzo della funzione \main è \TT{0x00401020} lanciamo il comando:

\begin{lstlisting}
tracer.exe -l:loops_2.exe bpf=loops_2.exe!0x00401020,trace:cc
\end{lstlisting}

\TT{BPF} sta per "set breakpoint on function".

Come risultato otteniamo gli script \TT{loops\_2.exe.idc} e \TT{loops\_2.exe\_clear.idc}.

\clearpage
Carichiamo \TT{loops\_2.exe.idc} in \IDA e vediamo:

\begin{figure}[H]
\centering
\myincludegraphics{patterns/09_loops/simple/IDA_tracer_cc.png}
\caption{\IDA con lo script .idc caricato}
\label{fig:loops_IDA_tracer}
\end{figure}

\ESI può essere da 2 a 9 all'inizio del corpo del loop, e da 3 a 0xA (10) dopo l'incremento.
Possiamo anche vedere che \main termina con 0 in \EAX..

\tracer genera anche il file \TT{loops\_2.exe.txt}, 
che contiene informazioni sui valori dei registri e su quante volte ogni istruzione è stata eseguita:

\lstinputlisting[caption=loops\_2.exe.txt]{patterns/09_loops/simple/loops_2.exe.txt}
\myindex{\GrepUsage}
Possiamo usare grep per trattare questo output.

}

\EN{\input{patterns/09_loops/simple/ARM/main_EN}}
\RU{\input{patterns/09_loops/simple/ARM/main_RU}}
\DE{\input{patterns/09_loops/simple/ARM/main_DE}}
\FR{\input{patterns/09_loops/simple/ARM/main_FR}}
\ITA{\subsubsection{ARM}

\myparagraph{\NonOptimizingKeilVI (\ARMMode)}

\lstinputlisting[label=Keil_number_sign,style=customasmARM]{patterns/09_loops/simple/ARM/Keil_ARM_O0.asm}

Il contatore di interazione $i$ viene memorizzato nel registro \Reg{4}.
L'istruzione \INS{MOV R4, \#2} inizializza $i$.
Le istruzioni \INS{MOV R0, R4} e \INS{BL printing\_function} rappresentano il corpo del ciclo, in cui la prima istruzione prepara
l'argomento per la funzione \ttf function e la seconda chiama la funzione.
\myindex{ARM!\Instructions!ADD}
L'istruzione  \INS{ADD R4, R4, \#1} aggiunge semplicemente 1 alla variabile $i$ ad ogni iterazione.
\myindex{ARM!\Instructions!CMP}
\myindex{ARM!\Instructions!BLT}
\INS{CMP R4, \#0xA} confronta $i$ con \TT{0xA} (10). 
L'istruzione successiva \INS{BLT} (\IT{Branch Less Than}) salta se $i$ è minotre di 10.
Altrimenti viene scritto 0 nel registro \Reg{0} (poiché la nostra funzione restituisce 0) e l'esecuzione delle funzione termina.

\myparagraph{\OptimizingKeilVI (\ThumbMode)}

\lstinputlisting[style=customasmARM]{patterns/09_loops/simple/ARM/Keil_thumb_O3.asm}

Praticamente uguale al precedente.

\myparagraph{\OptimizingXcodeIV (\ThumbTwoMode)}
\label{ARM_unrolled_loops}

\lstinputlisting[style=customasmARM]{patterns/09_loops/simple/ARM/xcode_thumb_O3.asm}

Questa era la mia funzione \ttf :

\begin{lstlisting}[style=customc]
void printing_function(int i)
{
    printf ("%d\n", i);
};
\end{lstlisting}

\myindex{Unrolled loop}
\myindex{Inline code}
LLVM non solo ha srotolato il loop ma ha anche "messo in linea" (\IT{inlined}) la funzione \ttf e inserito il suo corpo 8 volte
anziché chiamarla.

Ciò puà avvenire quando la funzione è molto semplice (come la mia) e quando non viene chiamata molto spesso (come in questo caso).

\myparagraph{ARM64: \Optimizing GCC 4.9.1}

\lstinputlisting[caption=\Optimizing GCC 4.9.1,style=customasmARM]{patterns/09_loops/simple/ARM/ARM64_GCC491_O3_EN.s}

\myparagraph{ARM64: \NonOptimizing GCC 4.9.1}

\lstinputlisting[caption=\NonOptimizing GCC 4.9.1 -fno-inline,style=customasmARM]{patterns/09_loops/simple/ARM/ARM64_GCC491_O3_EN.s}
}

\subsubsection{MIPS}

\ifdefined\RUSSIAN
\lstinputlisting[caption=\NonOptimizing GCC 4.4.5 (IDA),style=customasmMIPS]{patterns/09_loops/simple/MIPS_O0_IDA_RU.lst}

\myindex{MIPS!\Pseudoinstructions!B}
Новая для нас инструкция это \INS{B}. Вернее, это псевдоинструкция (\INS{BEQ}).
\fi

\ifdefined\ENGLISH
\lstinputlisting[caption=\NonOptimizing GCC 4.4.5 (IDA),style=customasmMIPS]{patterns/09_loops/simple/MIPS_O0_IDA_EN.lst}

\myindex{MIPS!\Pseudoinstructions!B}
The instruction that's new to us is \TT{B}. It is actually the pseudo instruction (\INS{BEQ}).
\fi

\ifdefined\FRENCH
\lstinputlisting[caption=\NonOptimizing GCC 4.4.5 (IDA),style=customasmMIPS]{patterns/09_loops/simple/MIPS_O0_IDA_FR.lst}

\myindex{MIPS!\Pseudoinstructions!B}
The instruction that's new to us is \TT{B}. It is actually the pseudo instruction (\INS{BEQ}).
L'instruction qui est nouvelle pour nous est \TT{B}. C'est une pseudo instruction (\INS{BEQ}).
\fi

\ifdefined\ITALIAN
\lstinputlisting[caption=\NonOptimizing GCC 4.4.5 (IDA),style=customasmMIPS]{patterns/09_loops/simple/MIPS_O0_IDA_EN.lst}

\myindex{MIPS!\Pseudoinstructions!B}
La nuova istruzione che incontriamo è \TT{B}. Di fatto di tratta della pseudo istruzione (\INS{BEQ}).
\fi


\subsubsection{One more thing}

Nel codice generato notiamo che dopo l'inizializzazione della variabile $i$ il corpo del loop non viene eseguito subito, ma
viene prima verificata la condizione per $i$ e solo dopo il loop è eseguito. 

Se la condizione non è soddisfatta all'inizio, il corpo del loop viene saltato.
Ciò è possibile nel seguente caso:

\lstinputlisting[style=customc]{patterns/09_loops/simple/loops_3_EN.c}

Se \IT{total\_entries\_to\_process} è 0, il corpo del loop non deve essere eseguito. Per questo motivo la condizione viene
controllata prima dell'esecuzione.

Un compilatore ottimizzante potrebbe comunque sostituire il controllo della condizione e il corpo del loop, se fosse sicuro dell'impossibilità 
della situazione qui descritta (come nel caso del nostro semplice esempio e i compilatori come Keil, Xcode (LLVM), MSVC in modalità ottimizzante).
