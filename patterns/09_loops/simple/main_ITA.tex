\subsection{Simple example}

% subsections
\input{patterns/09_loops/simple/x86}
\EN{\input{patterns/09_loops/cond_check/main_EN}}
\RU{\input{patterns/09_loops/cond_check/main_RU}}
\ITA{\subsection{Controllo della condizione}

E' importante tenere a mente che nel costrutto \IT{for()} la condizione non viene controllata alla fine, bensì all'inizio del ciclo,
prima dell'esecuzione del corpo.
Spesso però per il compilatore risulta più conveniente controllarla alla fine, dopo il corpo del loop.
A volte possono essere inseriti all'inizio anche altri controlli aggiuntivi.

Per esempio:

\lstinputlisting[style=customc]{patterns/09_loops/cond_check/1.c}

Optimizing GCC 5.4.0 x64:

\lstinputlisting[style=customasmx86]{patterns/09_loops/cond_check/1.s}

We see two checks.

\myindex{Hex-Rays}
Hex-Rays (versione 2.2.0) lo decompila così:

\lstinputlisting[style=customc]{patterns/09_loops/cond_check/hexrays.c}

In questo caso \IT{do/while()} può essere senza dubbio sostituito con \IT{for()}, e il primo controllo può essere rimosso.

}

\input{patterns/09_loops/simple/MIPS}

\subsubsection{One more thing}

Nel codice generato notiamo che dopo l'inizializzazione della variabile $i$ il corpo del loop non viene eseguito subito, ma
viene prima verificata la condizione per $i$ e solo dopo il loop è eseguito. 

Se la condizione non è soddisfatta all'inizio, il corpo del loop viene saltato.
Ciò è possibile nel seguente caso:

\lstinputlisting[style=customc]{patterns/09_loops/simple/loops_3_EN.c}

Se \IT{total\_entries\_to\_process} è 0, il corpo del loop non deve essere eseguito. Per questo motivo la condizione viene
controllata prima dell'esecuzione.

Un compilatore ottimizzante potrebbe comunque sostituire il controllo della condizione e il corpo del loop, se fosse sicuro dell'impossibilità 
della situazione qui descritta (come nel caso del nostro semplice esempio e i compilatori come Keil, Xcode (LLVM), MSVC in modalità ottimizzante).
