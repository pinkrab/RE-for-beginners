\subsubsection{MIPS}

\ifdefined\RUSSIAN
\lstinputlisting[caption=\NonOptimizing GCC 4.4.5 (IDA),style=customasmMIPS]{patterns/09_loops/simple/MIPS_O0_IDA_RU.lst}

\myindex{MIPS!\Pseudoinstructions!B}
Новая для нас инструкция это \INS{B}. Вернее, это псевдоинструкция (\INS{BEQ}).
\fi

\ifdefined\ENGLISH
\lstinputlisting[caption=\NonOptimizing GCC 4.4.5 (IDA),style=customasmMIPS]{patterns/09_loops/simple/MIPS_O0_IDA_EN.lst}

\myindex{MIPS!\Pseudoinstructions!B}
The instruction that's new to us is \TT{B}. It is actually the pseudo instruction (\INS{BEQ}).
\fi

\ifdefined\FRENCH
\lstinputlisting[caption=\NonOptimizing GCC 4.4.5 (IDA),style=customasmMIPS]{patterns/09_loops/simple/MIPS_O0_IDA_FR.lst}

\myindex{MIPS!\Pseudoinstructions!B}
The instruction that's new to us is \TT{B}. It is actually the pseudo instruction (\INS{BEQ}).
L'instruction qui est nouvelle pour nous est \TT{B}. C'est une pseudo instruction (\INS{BEQ}).
\fi

\ifdefined\ITALIAN
\lstinputlisting[caption=\NonOptimizing GCC 4.4.5 (IDA),style=customasmMIPS]{patterns/09_loops/simple/MIPS_O0_IDA_EN.lst}

\myindex{MIPS!\Pseudoinstructions!B}
La nuova istruzione che incontriamo è \TT{B}. Di fatto di tratta della pseudo istruzione (\INS{BEQ}).
\fi
