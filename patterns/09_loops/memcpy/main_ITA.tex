\subsection{Routine di copia della memoria a blocchi}
\label{loop_memcpy}

Le routine di copia della memoria reali possono copiare 4 o 8 byte ad ogni iterazione, usando \ac{SIMD}, 
vettorizzazione, etc.
Per semplicità questo esempio è il più semplice possibile.

\lstinputlisting[style=customc]{memcpy.c}

\subsubsection{implementazione diretta}

\lstinputlisting[caption=GCC 4.9 x64 ottimizzato per dimensione (-Os),style=customasmx86]{patterns/09_loops/memcpy/memcpy_GCC49_x64_Os_EN.s}

\lstinputlisting[caption=GCC 4.9 ARM64 ottimizzato per dimensione (-Os),style=customasmARM]{patterns/09_loops/memcpy/memcpy_GCC49_ARM64_Os_EN.s}

\lstinputlisting[caption=\OptimizingKeilVI (\ThumbMode),style=customasmARM]{patterns/09_loops/memcpy/memcpy_Keil_Thumb_O3_EN.s}

\subsubsection{ARM in ARM mode}

Keil in ARM mode sfrutta i suffissi condizioniali:

\lstinputlisting[caption=\OptimizingKeilVI (\ARMMode),style=customasmARM]{patterns/09_loops/memcpy/memcpy_Keil_ARM_O3_EN.s}

Per questo motivo c'è solo una sola istruzione di branch anziché due.

\subsubsection{MIPS}

\lstinputlisting[caption=GCC 4.4.5 optimized for size (-Os) (IDA),style=customasmMIPS]{patterns/09_loops/memcpy/memcpy_MIPS_Os_IDA_EN.lst}

\myindex{MIPS!\Instructions!LBU}
\myindex{MIPS!\Instructions!SB}

Qui incontriamo due nuove istruzioni : \INS{LBU} (\q{Load Byte Unsigned}) e \INS{SB} (\q{Store Byte}).

Come in ARM, tutti i registri MIPS sono larghi 32-bit e non ci sono sezioni più piccole come in x86.
Perciò anche dovendo operare con singoli byte è necessario allocare un intero registro a 32 bit.

\INS{LBU} carica un byte e azzera gli altri bit (\q{Unsigned}).
\myindex{MIPS!\Instructions!LB}

L'istruzione \INS{LB} (\q{Load Byte}) invece carica il byte ed estende il suo valore con segno a 32-bit.

\INS{SB} scrive un byte in memoria a partire dagli 8 bit bassi di un registro.

\subsubsection{vettorizzazione}

\GCC ottimizzante riesce a fare molto di più su questo esempio: \myref{vec_memcpy}.
