\subsubsection{x86}

\myparagraph{\NonOptimizing MSVC}

Dalla compilazione otteniamo:

\lstinputlisting[style=customasmx86]{patterns/10_strings/1_strlen/10_1_msvc_EN.asm}

\myindex{x86!\Instructions!MOVSX}
\myindex{x86!\Instructions!TEST}

Incontriamo due nuove istruzioni: \MOVSX e \TEST.

\label{MOVSX}
La prima ---\MOVSX--- prende un byte da un indirizzo di memoria e memorizza il valore in un registro a 32 bit. 
\MOVSX sta per \IT{MOV with Sign-Extend}. 
\MOVSX setta il resto dei bit, dall'8° al 31°, a 1 se il byte sorgente è \IT{negativo} o a 0 se è \IT{positivo}.

Segue il perchè.

Per default il tipo \Tchar è signed in MSVC e GCC. Se abbiamo due valori di cui uno \Tchar e l'altro 
\Tint, (anche \Tint è signed), il primo valore contenente -2 (codificato come \TT{0xFE}) e proviamo a compiare questo byte
nel contenitore \Tint, otteniamo \TT{0x000000FE}, che come intero con segno vale 254, non -2.
In signed int, -2 è codificato come \TT{0xFFFFFFFE}. 
Perciò se dobbiamo trasferire \TT{0xFE} da una variabile di tipo \Tchar a \Tint, dobbiamo identificare il suo segno e poi estenderlo.
Questo è ciò che fa \MOVSX.

Maggiori informazioni su questo argomento sono fornite nella sezione \q{\IT{\SignedNumbersSectionName}} (\myref{sec:signednumbers}).

E' difficile dire cosa succederebbe se il compilatore dovesse memorizzare una variabile \Tchar in \EDX. 
Potrebbe prendere solo una parte a 8 bit (per esempio \DL).
Apparentemente il \gls{register allocator} del compilatore funziona proprio così.

\myindex{ARM!\Instructions!TEST}

Successivamente vediamo \TT{TEST EDX, EDX}.
In questo caso controlla soltanto se il valore in \EDX è uguale a zero.
Maggiori informazioni sull'istruzione \TEST si trovano nella sezione dedicata ai bit fields~(\myref{sec:bitfields}).


\myparagraph{\NonOptimizing GCC}

Proviamo GCC 4.4.1:

\lstinputlisting[style=customasmx86]{patterns/10_strings/1_strlen/10_3_gcc.asm}

\label{movzx}
\myindex{x86!\Instructions!MOVZX}

Il risultato è pressoché identico a quello di MSVC, con l'uso di \MOVZX al posto di \MOVSX. 
\MOVZX sta per \IT{MOV with Zero-Extend}. 
Questa istruzione copia un valore a 8 bit o 16 bit in un registro a 32 bit, e setta il resto dei bit a 0.
E' conveniente in quanto rimpiazza la coppia di istruzioni:\\
\TT{xor eax, eax / mov al, [...]}.

Il compilatore avrebbe ovviamente potuto produrre il seguente codice: \\
\TT{mov al, byte ptr [eax] / test al, al}---il risultato è lo stesso, anche se in questo caso 
i bit alti del registro \EAX register conterranno rumore casuale. 
L'unico svantaggio è che questo codice è meno comprensibile, ma il compilatore non ha alcun obbligo di produrre codice comprensibile dagli umani.

\myindex{x86!\Instructions!SETcc}

L'altra nuova istruzione è \SETNZ.
Qui, se \AL non contiene zero, \TT{test al, al} 
setta il flag \ZF flag a 0, ma \SETNZ, se \TT{ZF==0} (\IT{NZ} sta per \IT{not zero}) setta \AL a 1.
In altre parole, \IT{se \AL non è zero, salta a loc\_80483F0}. 
Il compilatore ha emesso del codice ridondante, ma non dimentichiamoci che in questo caso le ottimizzazioni non sono attive.

\myparagraph{\Optimizing MSVC}
\label{strlen_MSVC_Ox}

Compiliamo ora l'esempio in MSVC 2012, attivando le ottimizzazioni (\Ox):

\lstinputlisting[caption=\Optimizing MSVC 2012 /Ob0,style=customasmx86]{patterns/10_strings/1_strlen/10_2_EN.asm}

Ora è tutto più semplice. Inutile dire che il compilatore è in grado di usare i registri con una simile efficienza solo
in caso di funzioni piccole con poche variabili locali.

\myindex{x86!\Instructions!INC}
\myindex{x86!\Instructions!DEC}
\INC/\DEC---sono istruzioni di \gls{incremento}/\gls{decremento}, in altre parole: aggiungono o sottraggono 1 a/da una variabile.

\input{patterns/10_strings/1_strlen/olly_EN.tex}

\myparagraph{\Optimizing GCC}

Vediamo cosa produce GCC 4.4.1 con le ottimizzazioni (\Othree key):

\lstinputlisting[style=customasmx86]{patterns/10_strings/1_strlen/10_3_gcc_O3.asm}

Quasi come MSVC, eccetto per la presenza di \MOVZX, che sarebbe potuta comunque essere sostituita da \\
\TT{mov dl, byte ptr [eax]}.

Forse per il generatore di codice di GCC è più semplice \IT{pensare} che l'intero registro a 32 bit \EDX sia allocato per
una variabile di tipo \Tchar, e assicurarsi che nei bit alti non ci sia mai alcun rumore.

\label{strlen_NOT_ADD}
\myindex{x86!\Instructions!NOT}
\myindex{x86!\Instructions!XOR}

Ancora più avanti incontriamo ancora una nuova istruzione ---\NOT. Questa istruzione inverte tutti i bit nell'operando. \\
Di fatto è un sinonimo dell'istruzione \TT{XOR ECX, 0ffffffffh}. 
\NOT e la successiva \ADD calcolano la differenza dei puntatori e sottraggono 1, soltanto in modo differente. 
All'inizio \ECX, in cui è memorizzato il puntatore a \IT{str}, viene invertito e decrementato di 1.

Vedi anche: \q{\SignedNumbersSectionName}~(\myref{sec:signednumbers}).
 
In altre parole, alla fine della funzione subito dopo il corpo del vengono eseguite queste operazioni:

\begin{lstlisting}[style=customc]
ecx=str;
eax=eos;
ecx=(-ecx)-1; 
eax=eax+ecx
return eax
\end{lstlisting}

\dots~ che è effettivamente equivalente a:

\begin{lstlisting}[style=customc]
ecx=str;
eax=eos;
eax=eax-ecx;
eax=eax-1;
return eax
\end{lstlisting}

Perchè GCC ha deciso che è meglio così? Difficile da dire, ma forse entrambe le varianti sono equivalenti in termini di efficienza.
