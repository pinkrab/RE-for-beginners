\clearpage
\myparagraph{\Optimizing MSVC + \olly}
\myindex{\olly}

Proviamo questo esempio (ottimizzato) in \olly. Questa è la prima iterazione:

\begin{figure}[H]
\centering
\myincludegraphics{patterns/10_strings/1_strlen/olly1.png}
\caption{\olly: inizio della prima iterazione}
\label{fig:strlen_olly_1}
\end{figure}

Vediamo che \olly ha trovato un loop e, per nostra comodità, lo ha racchiuso tra parentesi.
Facendo click di destra su \EAX e selezionando \q{Follow in Dump} la finestra della memoria si posizionerà in modo tale da osservare la memoria all'indirizzo contenuto in \EAX.
Vediamo infatti la stringa \q{hello!} in memoria. C'è almeno un byte zero dopo di essa, e poi robaccia casuale.
Se \olly vede un registro contenente un indirizzo valido che punta ad una stringa, visualizza la stringa stessa.

\clearpage
Premiamo F8 (\stepover) un po' di volte per arrivare all'inizio del corpo del ciclo:

\begin{figure}[H]
\centering
\myincludegraphics{patterns/10_strings/1_strlen/olly2.png}
\caption{\olly: inizio della seconda iterazione}
\label{fig:strlen_olly_2}
\end{figure}

Vediamo che \EAX contiene l'indirizzo del secondo carattere della stringa.

\clearpage


Dobbiamo premere F8 diverse volte per uscire dal ciclo:

\begin{figure}[H]
\centering
\myincludegraphics{patterns/10_strings/1_strlen/olly3.png}
\caption{\olly: sta per essere calcolata la differenza tra i puntatori}
\label{fig:strlen_olly_3}
\end{figure}

Notiamo che \EAX adesso contiene l'indirizzo del byte zero che si trova subito dopo la stringa.
Nel frattempo \EDX non è cambiato, e quindi punta ancora all'inizio della stringa.

La differenza tra questi due indirizzi sta per essere calcolata.

\clearpage
L'istruzione \SUB è appena stata eseguita:

\begin{figure}[H]
\centering
\myincludegraphics{patterns/10_strings/1_strlen/olly4.png}
\caption{\olly: \EAX sta per essere decrementato}
\label{fig:strlen_olly_4}
\end{figure}

La differenza tra i puntatori si trova ora nel registro \EAX---7.
Di fatto la lunghezza della stringa \q{hello!} è 6, ma includendo il terminatore (zero byte) è 7.
\TT{strlen()} deve restituire il numero di caratteri nella stringa escluso il terminatore, pertanto viene sottratto uno prima della fine
della funzione.
